%%%%%%%%%%%%%%%%%%%%%%%%%%%%%%%%%%%%%%%%%%%%%%%%%%%
%
%  New template code for TAMU Theses and Dissertations starting Fall 2012.
%  For more info about this template or the
%  TAMU LaTeX User's Group, see http://www.howdy.me/.
%
%  Author: Wendy Lynn Turner
%        Version 1.0
%  Last updated 8/5/2012
%
%%%%%%%%%%%%%%%%%%%%%%%%%%%%%%%%%%%%%%%%%%%%%%%%%%%

%%%%%%%%%%%%%%%%%%%%%%%%%%%%%%%%%%%%%%%%%%%%%%%%%%%%%%%%%%%%%%%%%%%%%%
%%                           SECTION I
%%%%%%%%%%%%%%%%%%%%%%%%%%%%%%%%%%%%%%%%%%%%%%%%%%%%%%%%%%%%%%%%%%%%%


\pagestyle{plain} % No headers, just page numbers
\pagenumbering{arabic} % Arabic numerals
\setcounter{page}{1}


\chapter{\uppercase {Introduction}}

%\section{Introduction}
\label{sec:intro}
Software Defined Networking (SDN) drastically changes the meaning and process of designing, building, testing, and operating networks. The core principle of the SDN paradigm is a separation of the network control and data planes. It enables network administrators to have a centralized view of the network and provides a standardized interface for remote configuration of network devices. In particular, the SDN approach provides an abstraction of the underlying data plane and an interface to manipulate that abstraction. This approach provides the capability to manage and operate a large network through a logically centralized controller and to define custom network behaviors.

The current support for wireless networking in SDN technologies has lagged behind its development and deployment for wired networks. The academic and industrial communities have focused primarily on wireline networks, while wireless networks have received significantly less attention.
Currently published SDN standards, the most popular of which is OpenFlow~\cite{openflow}, do not provide support for wireless protocols, which poses a major obstacle to developing SDN-enabled heterogeneous networks with wireless components. Attempts to support wireless networking within that framework have been ad hoc, and true network visibility is missing with respect to wireless protocols. The wireless protocols are also constantly changing and new protocols are being developed which have made the task for management of cross-layer characteristics in wireless radio networks incredibly difficult.

With such ever-changing wireless standards and protocols, there has been a conscious shift towards a programmatic approach for designing and implementing wireless radios. This has led to a tremendous interest in Software Defined Radios (SDR). SDR is a powerful concept in which filters, amplifiers, modulators and other complex signal processing blocks are realized in software, instead of on specialized hardware. As the task of signal processing is handed over to software, it is possible to use inexpensive general purpose hardware, connected to an RF front end, to create powerful and highly flexible radios.  

While the SDR paradigm has revolutionized the design of wireless radios, it does not provide an efficient method to control a network of SDRs. Since SDRs can be reconfigured to provide a wide variety of radio functionalities, there does not exist a consistent interface to expose the SDR's functional modules to the application developer. As modules can be added, removed or changed any time, the interface framework must be able to adapt to these changes. As such, the requisite framework should allow control of various constituent modules while hiding their complexity from the network operator. This level of abstraction is necessary because, as the network grows and becomes more heterogeneous, it is impossible for the operator to keep track of each and every wireless radio module. Here, by the notion of heterogeneous networks, we take into consideration a network containing both wired and wireless devices. Hence the architecture should enable network control, meet requirements of users and at the same time abstract away the details of the implementation.

The goal of this work is to fill the gap by extending the basic concepts of SDN to support wireless networks in a principled way. We also aim to integrate two key technologies, SDR and SDN, to provide a consistent interface to manage underlying abstractions of physical layer radio resources in SDRs. Note that any reasonable design must not be specific to a single protocol or implementation of SDN, but applicable to every viable implementation. Furthermore, any solution must not be tailored to a single application, but enable potentially \emph{any} application. Some examples are:

\begin{itemize}
\item \emph{Physical layer adaptation} including (i) \emph{frequency hopping} to resist narrowband interference and prevent unauthorized interception; (ii) \emph{transmission power control} to maintain a target link quality while reducing interference to other users and/or extending battery life; and (iii) \emph{adaptive modulation and coding} to trade-off throughput and communication reliability and adapt to channel conditions (e.g., pathloss and interference). %In Section~\ref{sec:evaluation}, we present an implementation of frequency hopping and adaptive modulation using the \crossflow framework.

\item \emph{Quality of service (QoS) provisioning} to provide QoS policies according to profiles implemented through medium access control, throttling, admission control, scheduling, and error control techniques (e.g., ARQ and FEC). This allows both coarse-grained and fine-grained QoS policies to be defined in the network.

\item \emph{Wireless handoff} to efficiently manage the Layer~2 transition of a client between APs (access points)

\item \emph{Client steering} optimizes the association of all the mobile stations in a wireless network area by directing a client to connect to specific APs based on signal strength and current usage.

\item \emph{Adaptive routing} to allow a distributed controller, with its global view of the network, to dynamically switch between existing proactive and reactive routing protocols, and novel software-defined routing protocols, depending on the network conditions and the application constraints.

% \item \emph{Adaptive routing} to allow a distributed \crossflow controller, with its global view of the network, to dynamically switch between reactive and proactive routing protocols depending on the network conditions and the application constraints.

\item \emph{Self healing network} to allow the controller to deploy fault management applications based upon self-healing mechanisms.

\item \emph{Cross-layer control} to allow joint optimization of parameters, algorithms, and protocols at all layers of the protocol stack.
\end{itemize}

To support a broad range of applications, our approach is to extend a generalized model of SDN derived from the OpenFlow specifications \cite{Casey:14}.  These extensions include support for wireless ports and channels as well as the events and counters specific to wireless networks. We also define various abstractions and their interfaces corresponding to a radio physical port, which is in line with the design principle for SDR. Our model enables SDN controllers to configure, query, and control IEEE~802.11 Access Points (APs) with respect to a wide range of wireless events and also allows fine-grained control of processing abilites of SDRs, which is independent of protocols being implemented. 

To validate the approach, we have implemented the model as an extension of the OpenFlow protocol, with a corresponding software implementation in the CPqD SoftSwitch software data plane \cite{ofsoftswitch13}. We refer to our extension of this model and our initial implementation in IEEE~802.11 APs as {\em \aetherflow} and the implementation of programmatic control of SDRs through GNU Radio \cite{gnuradio} framework as {\em \crossflow}. GNU Radio provides a modular and open-source digital signal processing environment for SDRs. The modules of GNU Radio are written in C++ and provide a mechanism to connect and manage data between them. A Python wrapper ties these blocks together to implement applications. We host GNU Radio on a Universal Software Radio Peripheral (USRP) E310 device from Ettus Research and also run CPqd SoftSwitch software~\cite{ofsoftswitch13} in a separate module as a switch agent.

We tested {\em \aetherflow} by developing a wireless mobility application that supports Layer~2 handoff of mobile stations between IEEE~802.11 access points. The resulting system performs on par with a traditional switching method. For testing {\em \crossflow}, develop two proof-of-concept applications, \emph{frequency hopping} and \emph{adaptive modulation}to validate our design principles mentuioned above.

Our contributions can be summarized as follows.

\begin{itemize}
\item The extension of the generic SDN model to provide explicit support for wireless radio interfaces and wireless access points.

\item An implementation of this extension based on the OpenFlow protocol and the CPqD SoftSwitch.

\item An implementation of a controller application using \aetherflow framework and experiments to demonstrate the viability of SDN-controlled access points for
efficient wireless handoff.

\item We provide proof-of-concept  applications using the \crossflow framework for design validation.
\end{itemize}

%The rest of the paper is organized as follows. In Section~\ref{sec:related} we review the related work done in this area. Section~\ref{sec:architecture} describes the \crossflow architecture with its SDN extensions. Section~\ref{sec:evaluation} describes a proof-of-concept implementation of two applications using our framework. Section~\ref{sec:conclusion} concludes the paper and discusses future work.
