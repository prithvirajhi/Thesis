\chapter{\uppercase {Related Work}}
\label{sec:related}
\section{Extension of Wireless LAN} 

The interest for extending WLAN capabilities has been a community goal for a long time, but traditional methods have certain constraints. For example, the approaches reported in \cite{murty10dyson,shrivastava09centaur} require modifications to the mobile clients (referred to as \emph{mobile stations} in the Wi-Fi standard), which makes those approaches hard to deploy and test.

A recent technical report by the Open Networking Foundation (ONF) \cite{onf13enabled} identified the challenges of mobile  networks, such as scalability, management, flexibility and cost, and provided a brief discussion of how SDN solutions can address these issues in few specific scenarios. A working group of Open Networking Foundation, Wireless \& Mobile Working Group (WMWG), has been focusing on devising new SDN architecture for wireless use cases of different types \cite{onf-wmwg:proposal}. However, to the best of our knowledge no concrete solutions were proposed by either ONF or WMWG up to now.

Several previous works presented systems that use OpenFlow extensions to achieve specific goals in wireless networks. In particular, OpenRoad \cite{yap10openroads, yap10blueprint, yap09stanford} proposes to use the OpenFlow framework as a research platform for Wi-Fi and Wi-MAX systems. The platform supports  slicing and virtualization of network resources, allowing different experimental  services to run at the same time. SoftCell \cite{jin13softcell} focused on LTE networks and proposed to integrate SDN framework into the LTE core network architecture.  The objective of \aetherflow is to design data plane control interfaces for wireless ports, which is different from these projects.

Other attempts to apply SDN to IEEE~802.11 networks include Odin~\cite{suresh12odin} and OpenSDWN~\cite{schulz15opensdwn}. They provide certain wireless interface control and configuration capabilities to the SDN controller. In these solutions, virtual access points and associated device contexts are created for each individual mobile device, and move across access points when the client handoff occurs. Such type of framework can handle user mobility gracefully, but results in overhead in terms of both computational load and traffic load during handoff, especially in the settings with a large number of clients and high user mobility. \aetherflow offers a set of interfaces that costs less but still supports a wide variety of wireless applications.

In contrast to the existing works, \aetherflow  provides a principled and general definition of wireless abstractions within an existing SDN framework. Our approach only requires incremental modifications to the existing SDN network elements.

\section{Programmatic Wireless Dataplane} 

The idea of providing a programmable wireless data plane has been implemented in ~\cite{openraio} and ~\cite{atomix}. Both these papers provide modular blocks and focus on real time guarantees for processing signals. But they do not provide any logical interface to control a network of such programmable devices. We choose GNU Radio in our design as it provides unlimited flexibility. Gudipati et al. ~\cite{softran} deals with centralized control of devices but focuses mainly on LTE networks. Our work is orthogonal to these works as we provide a mechanism for centralized control while making the exposed interfaces protocol independent.
The combination of SDRs and SDN has been introduced for various functionality in ~\cite{cho2014integration, sun2015integrating}, ~\cite{mancuso2014prototyping}, ~\cite{corbett2014countering} and ~\cite{sdnsdrinsno}. ~\cite{mancuso2014prototyping} deals with creation of testbed for LTE technologies while ~\cite{cho2014integration, sun2015integrating} focuses mainly on integration of SDN and SDR for 4G/5G technology.~\cite{gupta2015labview} describes a SDR model for management of interference in dense heterogeneous networks while ~\cite{corbett2014countering} developed a jamming architecture using SDN and SDR principles. ~\cite{sdnsdrinsno} provides a blueprint for LTE self-organizing networks(SONs) using SDN and SDR principles. These papers provide distinct solutions for various scenarios but do not provide a generic framework for handling various protocols in a principled manner.
