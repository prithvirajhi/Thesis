\section{\aetherflow Applications}
\label{sec:application}
The design of \aetherflow extends the capability of OpenFlow to wireless
(specifically IEEE~802.11) interfaces in a natural way. \aetherflow enables applications to control both wireline switches and wireless access points. 
%First, it enables an
%SDN controller to gain a more comprehensive control over wireless APs and mobile
%stations. Second, \aetherflow enables an SDN controller to control at the same
%time wireline switches and wireless access points in the network. 
As a result,
network applications that used to require different protocols and cooperation of
software from different vendors can now be implemented easily using the
\aetherflow framework. 

We use a Layer~2 fast handoff application to demonstrate the flexibility and new functionality offered by the \aetherflow framework. 
% enables more powerful network applications to be developed on SDN systems.
This application
aims to facilitate the process of mobile station handoff within the same subnet
during which a device's Layer~3 address is not changed.

% Traditional Layer~2 switched network learns mobility of a mobile client when it
% receives a packet from the client on a different port. This process is not
% completed and can lasts for a significant amount of time if no packet is sent by
% the client after the handoff. During the handoff process, any packet going to
% the client will be lost. In SDN solutions such as OpenFlow, the controller
% learns client mobility by receiving a packet-in message. However, A packet-in
% message cannot be triggered until the client sends a packet after handoff,
% resulting in the same problem as Layer~2 switched network.

% A Layer~2 fast handoff application is made possible by \aetherflow since it
% provides the SDN controller with wireless interface information, such as probe,
% association, client's signal strength, etc. With these information, when handoff
% occurs or even before that, the SDN controller may actively reconfigure
% switches in the network so that packets going to the client are redirected to
% the new AP that the client is associated to.

A typical Layer~2 fast handoff application runs in three phases. The
first phase is handoff prediction. The controller collects signal
strength information of the mobile stations by requesting statistics
of all mobile stations associated with APs under its control. At the same
time, it receives the probe signal strength of the mobile stations measured by
other APs from the probe event reports. By keeping these data updated in a
timely fashion, the controller may predict that a handoff is about to happen, e.g.
when the mobile station's signal strength to its associated AP gradually weakens
while the signal strength to another AP gradually strengthens. 

% A more accurate prediction is
% possible if history data and appropriate prediction model is used.

The second phase of the Layer~2 fast handoff application is multicasting. When a
handoff prediction of a mobile client is made, the controller multicasts all the
packets with the client as destination to both its current associated AP and the
predicted AP. The action is completed by
modifying the flow entries of the switches in the network. Multicast guarantees
that the client can receive packet immediately after it reassociates with the
new AP, thus minimizing the packet loss during the handoff.

The third phase is flow redirection. After the multicasting phase, if the client
associates with a new AP, the multicast is stopped and all the following packets
to the client will be redirected to the new AP. If the prediction is wrong and a
handoff did not occur within a certain timeout period, multicast is stopped and
all the following packets will be forwarded to the original AP that the client
is associated to. \aetherflow makes the decision possible with event report
interface that provides client association event report to the controller.

Other than Layer~2 handoff application, wireless network applications such as
client steering, user-based QoS control, etc.\ can also be easily implemented
using \aetherflow framework.
